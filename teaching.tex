\newpage
\null
\fancyhead[L]{\fontsize{12}{20}\sc joachim vandekerckhove}
\fancyhead[R]{\fontsize{12}{20}\sc teaching statement}
\pagenumbering{gobble}

\section*{Teaching statement}
\noindent I teach two types of courses.  First, there are lecture courses in which I consistently try to apply rational evidence-based approaches, such as:
\begin{itemize}
\item Setting clear goals on a weekly basis and referencing and reviewing these goals frequently
\item Visualization and live demonstration of the processes of experimental design and statistical analysis (“show and tell”)
\item Live (in-lab) practice and exercises, engaging students so that they attain individual ownership of a project
\item In-class demonstrations of methods to detect low-quality research – especially topical research with policy implications – and questionable research practices such as post-hoc subgroup analysis to engage students and fan critical attitudes
\item Questioning with randomized sampling to check for understanding
\item Copious and detailed feedback, both on an individual and plenary basis
\item Holding the highest expectations for students, but allowing each to learn at their own speed
\end{itemize}
\noindent Evidence-based teaching has been central to my educational strategy since the beginning of my career.  I care greatly about the benefit my students reap from their education, and I strongly believe that educational methods should be based on scientific evidence.  Every year, I observe the progress of my students by formal evaluations on multiple occasions, and every year the average and lowest grade improve dramatically during the quarter (on average, improvement tends to be on the order of a letter grade).\\

\noindent In the future, I expect to be teaching introductory undergraduate courses on statistical and experimental methods, and I intend to craft a curriculum that is based on first principles.  Starting from formal logic and some desirable properties of statistical decision making, I will guide students from logic to probability to evidence and inference.  I expect the class sizes to be larger than my current courses, which will necessitate slight changes in my teaching approach.  I will offer fewer live projects and fewer term papers but more frequent tests with fast turnaround.\\

\noindent The second type of course is the seminar-style graduate course, in which the focus is on reading recent publications and discussing their methods in depth.  Students are evaluated on their understanding of the material and -- when possible -- the integration of the new methods in their own projects.
