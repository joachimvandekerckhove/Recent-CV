\subsection*{Published work}\srefs
\item[52.] Mistry, P. K, Pothos, E. M., \vdkh{}, \& Trueblood, J. S. (in press). A quantum probability account of individual differences in causal reasoning. {\em Journal of Mathematical Psychology.}
\item[51.] Dutilh, G., Annis, J., Brown, S. D., Cassey, P., Evans, N. J., Grasman, R. P. P. P., Hawkins, G. E., Heathcote, A., Holmes, W. R., Krypotos, A., Kupitz, C. N., Leite, F. P., Lerche, V., Lin, Y., Logan, G. D., Palmeri, T. J., Starns, J. J., Trueblood, J. S., van Maanen, L., van Ravenzwaaij, D., \vdkh{}, Visser, I., Voss, A., White, C. N., Wiecki, T. V., Rieskamp, J., \& Donkin, C. (in press). The quality of response time data inference: A blinded, collaborative approach to the validity of cognitive models. {\em Psychonomic Bulletin \& Review.} Via osf.io/5tvse.
\item[50.] Heshmati, S., Oravecz, Z., Pressman, S., Batchelder, W. H., Muth, C., \& \vdkh{} (in press). What does it mean to feel loved? Cultural agreement and individual differences. {\em Journal of Social and Personal Relationships.}
\item[49.] Etz, A., Haaf, J. M., Rouder, J. N., \& \vdkh{} (2018). Bayesian inference and testing any hypothesis you can specify. {\em Advances in Methods and Practices in Psychological Science, 1,} 281--295. Via osf.io/htjgb.
\item[48.] \vdkh{}, Rouder, J. N., \& Kruschke, J. (2018). Editorial: Bayesian methods for advancing psychological science. {\em Psychonomic Bulletin \& Review, 25,} 1--4. Via osf.io/wx3h8.
\item[47.] Baribault, B., Donkin, C., Little, D. R., Trueblood, J. S., Oravecz, Z., van Ravenzwaaij, D., White, C. N., De Boeck, P., \& \vdkh{} (2018). Metastudies for robust tests of theory. {\em Proceedings of the National Academy of Sciences, 115,} 2607--2612. Via osf.io/g84py.
\item[46.] Rouder, J. N., Haaf, J. M., \& \vdkh{} (2018). Bayesian Inference in Psychology, Part IV: Parameter estimation and Bayes factors. {\em Psychonomic Bulletin \& Review, 25,} 102--113. Via osf.io/bvjg8.
\item[45.] Matzke, D., Boehm, U., \& \vdkh{} (2018). Bayesian Inference in Psychology, Part III: Bayesian parameter estimation in nonstandard models. {\em Psychonomic Bulletin \& Review, 25,} 77--101. Via osf.io/trhsy.
\item[44.] Etz, A., \& \vdkh{} (2018). Introduction to Bayesian inference for psychology. {\em Psychonomic Bulletin \& Review, 25,} 5--34. Via osf.io/n9mbu.
\item[43.] Okada, K., \vdkh{}, \& Lee, M. D. (2018). Modeling when people quit: Bayesian censored geometric models with hierarchical and latent-mixture extensions. {\em Behavior Research Methods, 50,} 406--415. Via osf.io/gytqz.
\item[42.] Dutilh, G., \vdkh{}, Ly, A., Matzke, D., Pedroni, A., Frey, R., Rieskamp, J., \& Wagenmakers, E. (2017). A test of the diffusion model explanation for the Worst Performance Rule using preregistration and blinding. {\em Attention, Perception, and Performance, 79,} 713--725.
\item[41.] van Ravenzwaaij, D., Donkin, C., \& \vdkh{} (2017). The EZ diffusion model provides a powerful test of simple empirical effects. {\em Psychonomic Bulletin \& Review, 24,} 547--556.
\item[40.] Lucio, P. S., Salum, G. A., Rohde, L. A. P., Gadelha, A., Swardfager, W., \vdkh{}, Pan, P. M., Polanczyk, G. V., do Rosario, M. C., Jackowski, A. P., Mari, J. d. J., \& Cogo-Moreira, H. (2017). Poor stimulus discriminability as a common neuropsychological deficit between ADHD and reading ability in young children: a  moderated mediation model. {\em Psychological Medicine, 47,} 255--266.
\item[39.] Nunez, M. D., \vdkh{}, \& Srinivasan, R. (2017). How attention influences perceptual decision making: Single-trial EEG correlates of drift-diffusion model parameters. {\em Journal of Mathematical Psychology, 76,} 117--130.
\item[38.] \vdkh{}, \& Wagenmakers, E. (2016). C. S. Peirce on the Crisis of Confidence and the "No More Bets" Heuristic. {\em The Winnower, 4843.}
\item[37.] Oravecz, Z., Muth, C., \& \vdkh{} (2016). Do people agree on what makes one feel loved? A cognitive psychometric approach to the consensus on felt love. {\em PLoS ONE, 11,} e0152803.
\item[36.] Etz, A., \& \vdkh{} (2016). A Bayesian perspective on the Reproducibility Project: Psychology. {\em PLoS ONE, 11,} e0149794.
\item[35.] Oravecz, Z., Tuerlinckx, F., \& \vdkh{} (2016). Bayesian data analysis with the bivariate hierarchical Ornstein-Uhlenbeck process model. {\em Multivariate Behavioral Research, 51,} 106--119.
\item[34.] Guan, M., \& \vdkh{} (2016). A Bayesian approach to mitigation of publication bias. {\em Psychonomic Bulletin \& Review, 23,} 74--86.
\item[33.] Oravecz, Z., Huentelman, M., \& \vdkh{} (2016). Sequential Bayesian updating for Big Data. {\em Big Data in Cognitive Science: From Methods to Insights,} pp.~13--33.
\item[32.] Van Elk, M., Matzke, D., Gronau, Q., Guan, M., \vdkh{}, \& Wagenmakers, E. (2015). Meta-analyses are no substitute for registered replications: a skeptical perspective on religious priming. {\em Frontiers in Psychology, 6,} 1365.
\item[31.] Kupitz, C. N., Buschkuehl, M., Jaeggi, S. M., Jonides, J., Shah, P., \& \vdkh{} (2015). A diffusion model account of the transfer-of-training effect. {\em Proceedings of the 37th Annual Conference of the Cognitive Science Society.}
\item[30.] Guan, M., Lee, M. D., \& \vdkh{} (2015). A hierarchical cognitive threshold model of human decision making on different length optimal stopping problems. {\em Proceedings of the 37th Annual Conference of the Cognitive Science Society.}
\item[29.] Mistry, P. K, Trueblood, J. S., \vdkh{}, \& Pothos, E. M. (2015). A latent-mixture quantum probability model of causal reasoning within a Bayesian inference framework. {\em Proceedings of the 37th Annual Conference of the Cognitive Science Society.}
\item[28.] Nunez, M. D., Srinivasan, R., \& \vdkh{} (2015). Individual differences in attention influence perceptual decision making. {\em Frontiers in Psychology, 6,} 18.
\item[27.] \vdkh{}, Matzke, D., \& Wagenmakers, E. (2015). Model comparison and the principle of parsimony. {\em Oxford Handbook of Computational and Mathematical Psychology,} pp.~300--317.
\item[26.] Zhang, S., Lee, M. D., \vdkh{}, Maris, G., \& Wagenmakers, E. (2014). Time-varying boundaries for diffusion models of decision making and response time. {\em Frontiers in Psychology, 5,} 1364.
\item[25.] Lee, M. D., Newell, B., \& \vdkh{} (2014). Modeling the adaptation of search termination in human decision making. {\em Decision, 1,} 223--251.
\item[24.] Murphy, P. R., \vdkh{}, \& Nieuwenhuis, S. (2014). Pupil-linked arousal determines variability in perceptual decision making. {\em PLOS Computational Biology, 10,} e1003854.
\item[23.] \vdkh{} (2014). A cognitive latent variable model for the simultaneous analysis of behavioral and personality data. {\em Journal of Mathematical Psychology, 60,} 58--71.
\item[22.] Wiech, K., \vdkh{}, Zaman, J., Tuerlinckx, F., Vlaeyen, J. W. S., \& Tracey, I. (2014). Influence of prior information on pain involves biased perceptual decision-making. {\em Current Biology, 24,} R679--R681.
\item[21.] Wabersich, D., \& \vdkh{} (2014). The RWiener package: an R package providing distribution functions for the Wiener diffusion model. {\em The R Journal, 6,} 49--56.
\item[20.] Oravecz, Z., \vdkh{}, \& Batchelder, W. H. (2014). Bayesian Cultural Consensus Theory. {\em Field Methods, 26,} 207--222.
\item[19.] Salum, G. A., Sergeant, J. A., Sonuga-Barke, E., \vdkh{}, Gadelha, A., Pan, P. M., Moriyama, T. S., Graeff-Martins, A. S., Gomes de Alvarenga, P., do Rosario, M. C., Manfro, G. G., Polanczyk, G. V., \& Rohde, L. A. P. (2014). Mechanisms underpinning inattention and hyperactivity: neurocognitive support for ADHD dimensionality. {\em Psychological Medicine, 44,} 3189--3201.
\item[18.] Wabersich, D., \& \vdkh{} (2014). Extending JAGS: A tutorial on adding custom distributions to JAGS (with a diffusion model example). {\em Behavior Research Methods, 46,} 15-28.
\item[17.] Salum, G. A., Sergeant, J. A., Sonuga-Barke, E., \vdkh{}, Gadelha, A., Pan, P. M., Moriyama, T. S., Graeff-Martins, A. S., Gomes de Alvarenga, P., do Rosario, M. C., Manfro, G. G., Polanczyk, G. V., \& Rohde, L. A. P. (2014). Specificity of basic information processing and inhibitory control in attention deficit/hyperactivity disorder. {\em Psychological Medicine, 44,} 617--631.
\item[16.] \vdkh{}, Guan, M., \& Styrcula, S. (2013). The consistency test may be too weak to be useful: Its systematic application would not improve effect size estimation in meta-analyses. {\em Journal of Mathematical Psychology, 57,} 170--173.
\item[15.] Pe, M., \vdkh{}, \& Kuppens, P. (2013). A diffusion model account of the relationship between the emotional flanker task and depression and rumination. {\em Emotion, 13,} 739--747.
\item[14.] Dutilh, G., Forstmann, B. U., \vdkh{}, \& Wagenmakers, E. (2013). A diffusion model account of age differences in posterror slowing. {\em Psychology and Aging, 28,} 64--76.
\item[13.] Dutilh, G., \vdkh{}, Forstmann, B. U., Keuleers, E., Brysbaert, M., \& Wagenmakers, E. (2012). Testing theories of post-error slowing. {\em Attention, Perception, \& Psychophysics, 7,} 454--465.
\item[12.] Oravecz, Z., Tuerlinckx, F., \& \vdkh{} (2011). A hierarchical latent stochastic differential equation model for affective dynamics. {\em Psychological Methods, 16,} 468--490.
\item[11.] \vdkh{}, Tuerlinckx, F., \& Lee, M. D. (2011). Hierarchical diffusion models for two-choice response times. {\em Psychological Methods, 16,} 44--62.
\item[10.] \vdkh{}, Verheyen, S., \& Tuerlinckx, F. (2010). A crossed random effects diffusion model for speeded semantic categorization data. {\em Acta Psychologica, 133,} 269--282.
\item[9.] Wetzels, R., \vdkh{}, Tuerlinckx, F., \& Wagenmakers, E. (2010). Bayesian parameter estimation in the Expectancy Valence model of the Iowa gambling task. {\em Journal of Mathematical Psychology, 54,} 14--27.
\item[8.] Dutilh, G., \vdkh{}, Tuerlinckx, F., \& Wagenmakers, E. (2009). A diffusion model decomposition of the practice effect. {\em Psychonomic Bulletin \& Review, 16,} 1026--1036.
\item[7.] Oravecz, Z., Tuerlinckx, F., \& \vdkh{} (2009). A hierarchical Ornstein-Uhlenbeck model for continuous repeated measurement data. {\em Psychometrika, 74,} 395--418.
\item[6.] Panis, S., De Winter, J., \vdkh{}, \& Wagemans, J. (2008). Identification of everyday objects on the basis of fragmented versions of outlines. {\em Perception, 37,} 271--289.
\item[5.] \vdkh{}, \& Tuerlinckx, F. (2008). Diffusion Model Analysis with MATLAB: A DMAT Primer. {\em Behavior Research Methods, 40,} 61--72.
\item[4.] \vdkh{}, Tuerlinckx, F., \& Lee, M. D. (2008). A Bayesian approach to diffusion process models of decision-making. {\em Proceedings of the 30th Annual Conference of the Cognitive Science Society,} pp.~1429--1434.
\item[3.] Spruyt, A., Hermans, D., De Houwer, J., \vdkh{}, \& Eelen, P. (2007). On the predictive validity of indirect attitude measures: Prediction of consumer choice behavior on the basis of affective priming in the picture--picture naming task. {\em Journal of Experimental Social Psychology, 43,} 599--610.
\item[2.] \vdkh{}, Panis, S., \& Wagemans, J. (2007). The concavity effect is a compound of local and global effects. {\em Perception \& Psychophysics, 69,} 1253--1260.
\item[1.] \vdkh{}, \& Tuerlinckx, F. (2007). Fitting the Ratcliff diffusion model to experimental data. {\em Psychonomic Bulletin \& Review, 14,} 1011--1026.
\erefs
