\newpage
\null
\fancyhead[L]{\fontsize{12}{20}\sc joachim vandekerckhove}
\fancyhead[R]{\fontsize{12}{20}\sc research statement}
\pagenumbering{gobble}

\section*{Research statement}
\noindent My main research interest is the development and adaptation of mathematical models of behavior and cognition as methods for data analysis. The principal goal of this line of work is to turn quantitative models into research tools with practical use.\\

\noindent This ultimate objective is typically operationalized through two concrete paths. The first path involves the construction of new models that address specific data analysis issues found in the literature. Two examples of such issues are efficient population-level analysis and nonindependence. Population-level analysis is made more efficient by implementing interindividual differences as statistical random effects, which serves the dual purpose of aligning the modeling practice with the reality of randomly sampled participants, as well as allowing the modeler to describe the distribution of parameters across participants and make probabilistic statements about intergroup differences. Nonindependence is a possibly more pervasive issue in which multiple measurements share a common cause. Examples range from subsequent reaction time measurements to long streams of electroencephalographic data. One strategy to treat this issue is to explicitly model a joint underlying cause -- which might be an adapting cognitive strategy in the case of sequential reaction times, or distributed neural activation in the case of EEG data.\\

\noindent The immediate goal of this line is to design models whose parameters tell us, as exactly and directly as possible, what we want to know -- while at the same time avoiding model misspecification. This path often includes collaboration with substantive researchers.\\

\noindent The second path focuses on model implementation: the development of statistical and computational methods for fitting and evaluating models. For much of this work, the major challenge is to develop algorithms, software, and statistics -- that is, the challenge is technical. For me, however, the most interesting part has always been the path from the set of assumptions that one, as a researcher, is willing to make about a psychological process and the (sometimes very) different set of assumptions that make up a formal model that can be applied in practice. It has been a guiding principle in my methodological work to make that road as short and unwinding as possible by increasing the flexibility of existing models and statistical frameworks. In the foreseeable future, my research program will continue to include work in the same directions of model building and model implementation.\\

\noindent In addition to this, I have taken up a separate research line in which I apply the techniques of behavioral modeling and modern statistics to the issue of publication bias and statistical forensics (i.e., methods to evaluate the quality of published analyses). This line of work is an interesting special case of my regular modeling work in which the subjects are researchers and publishers, and the behavioral processes are those that lead to scientific publications. In this line, the application of Bayesian statistics is particularly apposite since it often involves small amounts of data but significant prior information. The practical goal of this line of work is to use behavioral models and modern statistics in order to improve on meta-analysis by mitigating the damaging effects of such processes as publication bias and various questionable, but common, research practices.\\

\noindent The application of established psychometric, statistical, and computational methods in experimental contexts is the foundation of the road I expect my research career to follow. Combining these different disciplines remains a major untapped resource that has a large potential impact on data analysis practices in the field.